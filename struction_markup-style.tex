%----------------------------------------------------------------------------------------
% 标记的样式设定
%----------------------------------------------------------------------------------------

\newfontfamily\raintree{Raintree}[Scale=1.1] % 英文发布版带字体 注意版权问题

\newfontfamily\awesomefont{Font Awesome 5 Free Solid}[Scale=1.1] % 音乐符号

\newfontfamily\lmmonolightsmall{lmmonolt10-regular.otf}[Scale=1.1] % 拉大 1.1 倍适应中文正文
\newfontfamily\lmmonolightbig{lmmonolt10-regular.otf}[Scale=1.2] % 拉大 1.2 倍适应中文标题
\newfontfamily\lmmonocaps{lmmonocaps10-regular.otf}[Scale=1.1] % 拉大 1.2 倍适应中文

% NOTE: 废弃 仅作为 Archive
% \newcommand{\mt}{\raintree \fangsong} % mt -> mechinetalk NOTE: 另外一种样式

% mt ->  mechinetalk 需要区分中英文
% 中文:
\newcommand{\mtzh}{\lmmonolightsmall \fangsong} % \lmmonolight 1.1 倍放缩
\newcommand{\mtzhpr}[1]{{\mtzh #1}} % mtzhpr -> mtzh prefix
% 
% 英文:
\newcommand{\mten}{\raintree}
\newcommand{\mtenpr}[1]{{\mten #1}} % mtenpr -> mtzh prefix


\newcommand{\rt}{\sffamily} % rt -> radiotalk
% \newcommand{\py}{\freesans \sarasagothic} % py -> 怕音 % NOTE: 废弃 仅作为 Archive
\newcommand{\py}{\sffamily} % py -> 怕音

% \newcommand{\mtpr}[1]{{\mt #1}} % mtpr -> mechinetalk prefix % NOTE: 废弃 仅作为 Archive
\newcommand{\rtpr}[1]{{\rt #1}} % rtpr -> radiotalk prefix
\newcommand{\pypr}[1]{{\py #1}} % pypr -> py prefix
\newcommand{\rcpr}[1]{\emph{#1}} % 回忆 mrpr -> recall prefix
\newcommand{\thpr}[1]{\emph{#1}} % 心理活动 thpr -> thought prefix
\newcommand{\stpr}[1]{{\itshape \CJKunderdot{#1}}} % 强调 st -> stress % 不使用粗体 使用加点 % FIXME: 处理不优雅需要改进
\newcommand{\rspr}[1]{\emph{#1}} % 录音声音 rspr -> record sound
\newcommand{\wrpr}[1]{{\itshape #1}} % 书写 wr -> write 

%----------------------------------------------------------------------------------------
%	浮动体设置
%----------------------------------------------------------------------------------------

\usepackage{graphicx} % 插图
\graphicspath{{image/}} % 图片目录

\usepackage{float} % 用 H 选项来钉死浮动体

% 章节插图
\newcommand{\chapterintroimage}[1]{
\begin{figure}[H]
	\centering
	\includegraphics[width=0.9\linewidth]{#1}
\end{figure}
}


%----------------------------------------------------------------------------------------
%	环境标记
%----------------------------------------------------------------------------------------


% 题记样式
\newenvironment{intro}{\begin{center}\slshape}
{\end{center}\medskip}

% \newenvironment{song}{\begin{flushright}\itshape \kaishu {\awesomefont  \quad}}{\end{flushright}}

% 唱歌样式
\newenvironment{song}{\begin{flushright}\itshape}{\end{flushright}}

% 音乐样式
\newenvironment{music}{\begin{flushright}{\awesomefont\normalfont  }\hspace{\fill}\itshape}{\end{flushright}}

% \newenvironment{song}{\begin{flushright}\itshape \kaishu \marginpar[]{ \awesomefont }}{\end{flushright}}

% \newenvironment{song}{\marginpar[]{ \awesomefont }\begin{flushright}\itshape \kaishu}{\end{flushright}}


\newcommand\daytimeplace[4]{
  \begin{table}[H]
    \centering \fangsong \lmmonolightbig
    第 #1 天

    大约时间:#2

    地点:#3

    {\lmmonolightsmall {\lmmonocaps Loaction}: #4}
  \end{table}
}

\newcommand\unknowndaytimeplace{
  \begin{table}[H]
    \centering \fangsong \lmmonolightbig
    日期:N/A

    时间:N/A
    
    地点:N/A
  \end{table}
}

\newcommand\englishdaytimeplace[3]{
  \begin{table}[H]
    \centering \ttfamily
    \textsc{Day} #1

    \textsc{Time} approximately #2

    \textsc{Loaction}: #3
  \end{table}
}

\newcommand\englishunknowndaytimeplace{
  \begin{table}[H] % 钉死
    \centering
    \begin{tabular}{lr}
      \texttt{\textsc{Day}} & \texttt{N.A.} \\
      \texttt{\textsc{Time}} & \texttt{N.A.} \\
      \texttt{\textsc{Loaction}} & \texttt{N.A.} \\
    \end{tabular}    
  \end{table}
}

\font\eightssi=cmssqi8 % PostScript 字体

\newenvironment{motto}{
~\vfill
\begin{flushright}
  \eightssi \fangsong
}{\end{flushright}}

\newenvironment{note}{
\paragraph{尾注}}{\bigskip}

\newenvironment{engnote}{
\paragraph{Tailnote}}{\bigskip}

\newcommand\printstatus[7]{
\begin{flushleft}
\textbf{帕比的属性}
\end{flushleft}

\begin{itemize}
\item 力量:#1
\item 感知:#2
\item 耐力:#3
\item 魅力:#4
\item 智力:#5
\item 敏捷:#6
\item 幸运:#7
\end{itemize}
}

\newcommand\engprintstatus[7]{
\begin{flushleft}
\textbf{Puppy's S.P.E.C.I.A.L.}
\end{flushleft}

\begin{itemize}
\item{\makebox[2cm][l]{Strength:} #1}
\item{\makebox[2cm][l]{Perception:} #2}
\item{\makebox[2cm][l]{Endurance:} #3}
\item{\makebox[2cm][l]{Charisma:} #4}
\item{\makebox[2cm][l]{Intelligence:} #5}
\item{\makebox[2cm][l]{Agility:} #6}
\item{\makebox[2cm][l]{Luck:} #7}
\end{itemize}
}

%----------------------------------------------------------------------------------------
% 页眉样式
%----------------------------------------------------------------------------------------

% 在载入 geometry 之后

\usepackage{fancyhdr}

\fancypagestyle{chinese}{%
\fancyhf{} % 清空

\fancyhead[LE,RO]{\thepage}

\fancyhead[LO]{\sffamily {\leftmark}} % 奇数页  \leftmark = 章节名称
\fancyhead[RE]{\sffamily {辐射小马国:粉色双眸}} % 偶数页
}

\fancypagestyle{english}{%
\fancyhf{} % 清空

\fancyhead[LE,RO]{\thepage}

\fancyhead[LO]{\sffamily {\leftmark}} % 奇数页  \leftmark = 章节名称
\fancyhead[RE]{\sffamily \uppercase{Fallout Equestria: Pink Eyes}} % 偶数页
}

\newcommand{\fancyheadmanuallyenglish}[1]{
  \fancyhf{} % 清空

  \fancyhead[LE,RO]{\thepage}
  
  \fancyhead[LO]{\sffamily \uppercase{#1}} % 奇数页  \leftmark = 章节名称
  \fancyhead[RE]{\sffamily \uppercase{Fallout Equestria: Pink Eyes}} % 偶数页
}

\newcommand{\fancyheadmanuallychinese}[1]{
  \fancyhf{} % 清空

  \fancyhead[LE,RO]{\thepage}
  
  \fancyhead[LO]{\sffamily {#1}} % 奇数页  \leftmark = 章节名称
  \fancyhead[RE]{\sffamily {辐射小马国:粉色双眸}} % 偶数页
}

%----------------------------------------------------------------------------------------
%	分割线样式
%----------------------------------------------------------------------------------------

% 第一种样式
% \newcommand{\horizonline}{
%     \begin{center}\rule{0.5\linewidth}{\linethickness}\end{center}
% }

% 第二种样式
% \newcommand{\horizonline}{
%     \begin{center}\includegraphics[width=0.5\linewidth]{image_line.png}\end{center}
% }

% 第三种样式 via: fancyhdr 宏包手册
\usepackage{fourier-orns}
\newcommand{\horizonline}{%
  \begin{center}
    \rule{0.2\linewidth}{\linethickness}
    \raisebox{-2.1pt}{\hspace{1em}\decofourleft\decotwo\decofourright\hspace{1em}}\,   % 然而并不对称 插入 \, 先修补一下
    \rule{0.2\linewidth}{\linethickness}
  \end{center}
}


%----------------------------------------------------------------------------------------
% 脚注样式
%----------------------------------------------------------------------------------------

% \usepackage{xpatch} % 提供 \xpatchcmd NOTE: 已在 markup-hack 中加载

% 调整脚注的分割线 via:https://github.com/muzimuzhi/latex-examples/blob/master/footnote-chinese-style.tex https://zhuanlan.zhihu.com/p/74515148

\xpatchcmd\footnoterule
  {.4\columnwidth}
  {1in}
  {}{\fail}

% ----------


\newfontfamily\footnotefont{Noto Serif CJK SC}[Script=CJK Ideographic,Language=Chinese Simplified]
% \newCJKfontfamily[footnotecjk]\footnotecjkfont{Noto Serif CJK SC}[Script=CJK Ideographic,Language=Chinese Simplified]

\newcommand{\footnotespacefix}{\hspace{-0.5\ccwd}}

\usepackage{scrextend} % 提供 \deffootnote

% NOTE: 脚注大小 7.5bp 空两个字 2\ccwd
% ATTENTION: 不要使用 em 作为单位 因为英文字体已经被放大 1.1 倍
% NOTE: 直接使用中文字体

% 样式:[1]

% \renewcommand{\thefootnote}{{\footnotefont[\arabic{footnote}]}}
% \deffootnote[0em]{0em}{2\ccwd}{{\footnotesize\thefootnotemark\hspace{\ccwd}}}

% 样式:〔1〕(最好)

\renewcommand{\thefootnote}{{\footnotefont\makexeCJKinactive\hspace{-0.5\ccwd}〔\arabic{footnote}〕\hspace{-0.5\ccwd}\makexeCJKactive}}
\deffootnote[0em]{0em}{2\ccwd}{{\footnotesize\thefootnotemark\hspace{\ccwd}}}

% 样式:[1]

% \renewcommand{\thefootnote}{{\footnotefont\makexeCJKinactive\hspace{-0.5\ccwd}[\arabic{footnote}]\hspace{-0.5\ccwd}\makexeCJKactive}}
% \deffootnote[0em]{0em}{2\ccwd}{{\footnotesize\thefootnotemark\hspace{\ccwd}}}

% FIXME: hyperref 给的方框过大
% FIXME: 需要禁止脚注换行

% 样式:带圈数字
% via: [带圈数字](https://stone-zeng.github.io/2019-02-09-circled-numbers)

% \usepackage{xunicode-addon}

% % XeLaTeX 下需要把全体带圈数字都设置成 Default 类
% \xeCJKDeclareCharClass{Default}{"24EA, "2460->"2473, "3251->"32BF}

% % 放置钩子,只让带圈字符才需更换字体

% \AtBeginUTFCommand[\textcircled]{\begingroup\footnotefont}
% \AtEndUTFCommand[\textcircled]{\endgroup}

% \renewcommand{\thefootnote}{\textcircled{\arabic{footnote}}}
% \deffootnote[0em]{0em}{2\ccwd}{\footnotesize\thefootnotemark\hspace{\ccwd}} % 需要留空


% NOTE: 以下也可行 via:https://github.com/muzimuzhi/latex-examples/blob/master/footnote-chinese-style.tex

% \makeatletter
% % add wrapper \textcircled,
% % adapted from latex.ltx, line 6380:
% \renewcommand\thefootnote{\textcircled{\@arabic\c@footnote}}

% % use separate footnote mark command
% \xpatchcmd\@makefntext
%   {\hb@xt@1.8em{\hss\@makefnmark}}
%   {\hb@xt@1.8em{\hss\@makefnmarkNormal}\space}
%   {}{\fail}

% % use non-suprescript style with lower baseline
% % adapted from latex.ltx, line 6383:
% % \def\@makefnmarkNormal{\lower .3ex \hbox{\normalfont\@thefnmark}} % NOTE: 不下调基线
% \def\@makefnmarkNormal{\hbox{\normalfont\@thefnmark}}

% \makeatother
