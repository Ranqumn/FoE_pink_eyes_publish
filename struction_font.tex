%----------------------------------------------------------------------------------------
% 将 U+2014 破折号合字 暴力修正

% via: https://github.com/CTeX-org/ctex-kit/issues/382#issuecomment-430873626

\ExplSyntaxOn
\xeCJK_new_class:n { PoZheHao }
\__xeCJK_save_CJK_class:n { PoZheHao }
\xeCJK_declare_char_class:nn { PoZheHao } { "2014 }
\seq_map_inline:Nn \g__xeCJK_class_seq
  {
    \str_if_eq:nnF {#1} { PoZheHao }
      {
        \xeCJK_copy_inter_class_toks:nnnn { PoZheHao } {#1} { FullRight } {#1}
        \xeCJK_copy_inter_class_toks:nnnn {#1} { PoZheHao } {#1} { FullRight }
      }
  }

% NOTE: 解决「。——」这种情况报错问题
% NOTE: 或者可以使用手动(正则表达式)放置空箱 `。\mbox{}——` 来解决
% 
% via: https://github.com/CTeX-org/ctex-kit/issues/382#issuecomment-491951413

\prg_set_conditional:Npnn \__xeCJK_punct_if_right:N #1 { p , T , F , TF }
  {
    \if_int_compare:w \xeCJK_token_value_class:N #1 =
                      \xeCJK_class_num:n { FullRight }
      \prg_return_true:
    \else:
      \if_int_compare:w \xeCJK_token_value_class:N #1 =
                        \xeCJK_class_num:n { PoZheHao }
        \prg_return_true:
      \else:
        \prg_return_false:
      \fi:
    \fi:
  }
\ExplSyntaxOff

% 有 0.5% 的过大误差 就先这样

% NOTE: 对 思源系字体 一定要开启特性
% 比如 `\setCJKmainfont{Noto Sans CJK SC}[Script=CJK Ideographic,Language=Chinese Simplified]`
% 或者开启 fwid 特性 `[RawFeature = +fwid]`


%----------------------------------------------------------------------------------------
%	Fonts
%----------------------------------------------------------------------------------------

% 使用 Noto CJK

% \newCJKfontfamily[notoserifcjksc]\notoserif{Noto Serif CJK SC}
% \newCJKfontfamily[sarasagothicsc]\sarasagothic{Sarasa Gothic SC}


% \newfontfamily\freesans{FreeSans}
% \newfontfamily\lmmono{lmmono10-regular.otf} % Latin Modern Mono 默认的等距字体 NOTE: 该字体不区分前后引号

% \setmainfont{Liberation Serif}[
%   UprightFont       = *,
%   ItalicFont        = * Italic,
%   BoldFont          = * Bold,
%   BoldItalicFont    = * Bold Italic,
%   Scale             = 1.1
% ]

\setmainfont{Libertinus Serif}[
  UprightFont       = *,
  ItalicFont        = * Italic,
  BoldFont          = * Semibold,
  BoldItalicFont    = * Semibold Italic,
  Scale             = 1.1
]

% \setmainfont{Linux Libertine O}[
%   UprightFont       = *,
%   ItalicFont        = * Italic,
%   BoldFont          = * Semibold,
%   BoldItalicFont    = * Semibold Italic,
%   Scale             = 1.1
% ]

% Noto Serif 可用

% FIXME: 切换成 Noto Serif 脚注不能工作 原因未知

% \setmainfont{Noto Serif}[
%   UprightFont       = * Light,
%   ItalicFont        = * Light Italic,
%   BoldFont          = * SemiBold,
%   BoldItalicFont    = * SemiBold Italic,
%   Scale             = 1.1
% ]

% \setmainfont{Source Serif Pro}[
%   UprightFont       = *,
%   ItalicFont        = * Italic,
%   BoldFont          = * Semibold,
%   BoldItalicFont    = * Semibold Italic,
%   Scale             = 1.1
% ]

% \setmainfont{lmroman}[
%   Extension         = .otf,
%   UprightFont       = *10-regular,
%   ItalicFont        = *10-italic,
%   BoldFont          = *10-bold,
%   BoldItalicFont    = *10-bolditalic,
%   SmallCapsFont     = *caps10-regular,
%   SlantedFont       = *slant10-regular,
%   Scale             = 1.1
% ]

\setsansfont{lmsans10}[
  Extension         = .otf,
  UprightFont       = *-regular,
  ItalicFont        = *-oblique,
  BoldFont          = *-bold,
  BoldItalicFont    = *-boldoblique,
  Scale             = 1.1
]

% \setsansfont{Source Sans Pro}[
%   UprightFont       = * Regular,
%   ItalicFont        = * Italic,
%   BoldFont          = * Semibold,
%   BoldItalicFont    = * Semibold Italic,
%   Scale             = 1.1
% ]


\setmonofont{lmmono}[
  Extension         = .otf,
  UprightFont       = *10-regular,
  ItalicFont        = *10-italic,
  SmallCapsFont     = *caps10-regular,
  Scale             = 1.1
]

\setCJKmainfont{Noto Serif CJK SC}[
  UprightFont       = *,
  BoldFont          = * Bold,
  ItalicFont        = FandolKai-Regular,
  Script            = CJK Ideographic,
  Language          = Chinese Simplified
]

% NOTE: FandolKai 没有 Script 选项 
% ATTENTION: FandolKai 在调用时需要与 Noto Serif CJK SC 在同一个目录下

\setCJKsansfont{Noto Sans CJK SC}[
  UprightFont       = * DemiLight,
  BoldFont          = * Medium,
  Script            = CJK Ideographic,
  Language          = Chinese Simplified
]

\setCJKmonofont{FandolFang-Regular.otf}

% via: CTeX 宏集的默认定义
% NOTE: 放在暴力修正 Fandol 字体之前

\newCJKfontfamily[zhfs]\fangsong{FandolFang-Regular.otf}
\newCJKfontfamily[zhkai]\kaishu{FandolKai-Regular.otf}
\newCJKfontfamily[zhhei]\heiti{FandolHei-Regular.otf}

% 暴力修正 FandolKai Fandol-FangSong 的基线
% via: https://github.com/clerkma/ptex-ng-dist/issues/5

% \usepackage{etoolbox} % 提供 \appto NOTE: markup-style 中已载入

\makeatletter
\newcommand*\original@CJKsymbol{}
\newcommand*\original@CJKpunctsymbol{}
\let\original@CJKsymbol\CJKsymbol
\let\original@CJKpunctsymbol\CJKpunctsymbol
\newcommand*\raise@Fandol@CJK[1]{\raise0.08\ccwd\hbox{#1}}
\appto\itshape{%
  \let\CJKsymbol\raise@Fandol@CJK
  \let\CJKpunctsymbol\raise@Fandol@CJK
}
\appto\kaishu{%
  \let\CJKsymbol\raise@Fandol@CJK
  \let\CJKpunctsymbol\raise@Fandol@CJK
}
\appto\fangsong{%
  \let\CJKsymbol\raise@Fandol@CJK
  \let\CJKpunctsymbol\raise@Fandol@CJK
}
\appto\upshape{%
  \let\CJKsymbol\original@CJKsymbol
  \let\CJKpunctsymbol\original@CJKpunctsymbol
}
% 修正包裹在 Fandol 字体中的脚注 另见 latex.ltx 6403 行
\appto\reset@font{%
  \let\CJKsymbol\original@CJKsymbol
  \let\CJKpunctsymbol\original@CJKpunctsymbol
}
\makeatother

