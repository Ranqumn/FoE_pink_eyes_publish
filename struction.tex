%----------------------------------------------------------------------------------------
%	宏包设置
%----------------------------------------------------------------------------------------


% 浮动体

\usepackage{graphicx} % 插图
\graphicspath{{image/}} % 图片目录

\usepackage[section]{placeins} % 防止浮动体跨区

% --------

% 排版专用

\usepackage{nag} % 检查过时指令

\usepackage[english]{babel}
\usepackage{microtype} % 微排版

\usepackage{fancyhdr} % 修改页眉


% \usepackage{scrextend} 
% % 修改脚注样式
% \deffootnote[1.5em]{1.5em}{1em}{〔\thefootnotemark〕\space}
% \deffootnotemark{\textsuperscript{〔\thefootnotemark〕}}
% NOTE: 还要调整基线


%----------------------------------------------------------------------------------------
%	Input and Fonts
%----------------------------------------------------------------------------------------

% 使用 XeLaTeX 
\usepackage{xltxtra,xunicode}

\usepackage{textcomp} % 扩展特殊字符

% 使用 Noto CJK

\newCJKfontfamily[notoserifcjksc]\notoserif{Noto Serif CJK SC}
\newCJKfontfamily[notosanscjksc]\notosans{Noto Sans CJK SC}
\newCJKfontfamily[sarasagothicsc]\sarasagothic{Sarasa Gothic SC}


\newfontfamily\wnotoserif{Noto Serif CJK SC}
\newfontfamily\wnotosans{Noto Sans CJK SC}
\newfontfamily\wsarasagothic{Sarasa Gothic SC}

% 注意西文字体需要更换

\setCJKmainfont{Noto Serif CJK SC}

% 因为有些字 Fandol 没有 所以使用 Noto Serif
% 这是一个不优雅的处理,先这样

\renewcommand{\emph}[1]{{\em \kaishu #1}}
\newcommand{\mt}{\ttfamily \fangsong} % mt -> mechinetalk
\newcommand{\rt}{\sffamily \notosans} % rt -> radiotalk
\newcommand{\py}[1]{{\wsarasagothic \sarasagothic #1}} % py -> 怕音

%----------------------------------------------------------------------------------------
%	数学相关
%----------------------------------------------------------------------------------------

\usepackage{amsmath,amssymb,amsthm}
\usepackage{siunitx}

%----------------------------------------------------------------------------------------
%	LINKS
%----------------------------------------------------------------------------------------

\usepackage{url}
\usepackage[setpagesize=false,unicode=false,xetex]{hyperref}

\hypersetup{breaklinks=true,
            pdfauthor={RandomQuantum},
            pdftitle={辐射小马国:粉色双眸}
            }

%----------------------------------------------------------------------------------------
%	环境
%----------------------------------------------------------------------------------------

\newenvironment{intro}{\begin{flushleft}\slshape \kaishu}
{\end{flushleft}\medskip}

\newenvironment{song}{\begin{flushright}\itshape \kaishu}{\end{flushright}}

\newcommand\daytimeplace[4]{
  \begin{center}
    \ttfamily
    第 #1 天

    大约时间:#2

    地点:#3

    LOCATION: #4
  \end{center}
}

\newcommand\unknowndaytimeplace{
  \begin{center}
    \ttfamily
    日期:NaN

    时间:NaN
    
    地点:NaN
  \end{center}
}

\newcommand\englishdaytimeplace[3]{
  \begin{center}
    \ttfamily
    Day #1

    TIME approximately #2

    LOCATION: #3
  \end{center}
}

\newcommand\englishunknowndaytimeplace{
  \begin{center}
    \ttfamily
    DAY N.A.

    TIME N.A.
    
    LOCATION N.A.
  \end{center}
}

\font\eightssi=cmssqi8

\newenvironment{motto}{
~\vfill
\begin{flushright}
  \eightssi \fangsong
}{\end{flushright}}

\newenvironment{note}{
\paragraph{蹄注}}{\bigskip}

\newenvironment{engnote}{
\paragraph{Hoovenote}}{\bigskip}

\newcommand\printstatus[7]{
\begin{flushleft}
\textbf{帕比的属性}
\end{flushleft}

\begin{itemize}
\item 力量:#1
\item 感知:#2
\item 耐力:#3
\item 魅力:#4
\item 智力:#5
\item 敏捷:#6
\item 幸运:#7
\end{itemize}
}

\newcommand\engprintstatus[7]{
\begin{flushleft}
\textbf{Puppy's S.P.E.C.I.A.L.}
\end{flushleft}

\begin{itemize}
\item Strength: #1
\item Perception: #2
\item Endurance: #3
\item Charisma: #4
\item Intelligence: #5
\item Agility: #6
\item Luck: #7
\end{itemize}
}


%----------------------------------------------------------------------------------------
%	目录
%----------------------------------------------------------------------------------------

\setcounter{secnumdepth}{0} % 取消节编号


%----------------------------------------------------------------------------------------
%	定义符号与缩写
%----------------------------------------------------------------------------------------

% 第一种样式
% \newcommand{\horizonline}{
%     \begin{center}\rule{0.5\linewidth}{\linethickness}\end{center}
% }

% 第二种样式
\newcommand{\horizonline}{
    \begin{center}\includegraphics[width=0.5\linewidth]{image_line.png}\end{center}
}



