%----------------------------------------------------------------------------------------
% 将 U+2014 破折号合字 暴力修正

% via: https://github.com/CTeX-org/ctex-kit/issues/382#issuecomment-430873626

\ExplSyntaxOn
\xeCJK_new_class:n { PoZheHao }
\__xeCJK_save_CJK_class:n { PoZheHao }
\xeCJK_declare_char_class:nn { PoZheHao } { "2014 }
\seq_map_inline:Nn \g__xeCJK_class_seq
  {
    \str_if_eq:nnF {#1} { PoZheHao }
      {
        \xeCJK_copy_inter_class_toks:nnnn { PoZheHao } {#1} { FullRight } {#1}
        \xeCJK_copy_inter_class_toks:nnnn {#1} { PoZheHao } {#1} { FullRight }
      }
  }

% via: https://github.com/CTeX-org/ctex-kit/issues/382#issuecomment-491951413

\prg_set_conditional:Npnn \__xeCJK_punct_if_right:N #1 { p , T , F , TF }
  {
    \if_int_compare:w \xeCJK_token_value_class:N #1 =
                      \xeCJK_class_num:n { FullRight }
      \prg_return_true:
    \else:
      \if_int_compare:w \xeCJK_token_value_class:N #1 =
                        \xeCJK_class_num:n { PoZheHao }
        \prg_return_true:
      \else:
        \prg_return_false:
      \fi:
    \fi:
  }
\ExplSyntaxOff

% 有 0.5% 的过大误差 就先这样

% NOTE: 对 思源系字体 一定要开启特性
% 比如 `\setCJKmainfont{Noto Sans CJK SC}[Script=CJK Ideographic,Language=Chinese Simplified]`

\XeTeXgenerateactualtext=1 % 让复制文本正确


%----------------------------------------------------------------------------------------
%	宏包设置
%----------------------------------------------------------------------------------------

\usepackage{nag} % 检查过时指令

% 浮动体

\usepackage{graphicx} % 插图
\graphicspath{{image/}} % 图片目录

% \usepackage[section]{placeins} % 防止浮动体跨区 NOTE: 用 float 来实现
\usepackage{float} % 用 H 选项来钉死浮动体

% \usepackage{calc} % 计算 NOTE: 最好还是手动计算 为避免误差

%----------------------------------------------------------------------------------------
%	Input and Fonts
%----------------------------------------------------------------------------------------

% 使用 XeLaTeX 
\usepackage{xltxtra,xunicode}

\usepackage{textcomp} % 扩展特殊字符

\defaultfontfeatures{Ligatures=TeX} % fontspec 使用 TeX 的语法打特殊字符

% 使用 Noto CJK


\newfontfamily\awesomefont{Font Awesome 5 Free Solid}[Scale=1.1]

\setmainfont{lmroman10-regular.otf}[
  ItalicFont=lmroman10-italic.otf,
  BoldFont=lmroman10-bold.otf,
  BoldItalicFont=lmroman10-bolditalic.otf,
  SmallCapsFont=lmromancaps10-regular.otf,
  SlantedFont=lmromanslant10-regular.otf,
  Scale=1.1
]

\setsansfont{lmsans10-regular.otf}[
  ItalicFont=lmsans10-oblique.otf,
  BoldFont=lmsans10-bold.otf,
  BoldItalicFont=lmsans10-boldoblique.otf,
  Scale=1.1
]

\setmonofont{lmmono10-regular.otf}[
  ItalicFont=lmmono10-italic.otf,
  SmallCapsFont=lmmonocaps10-regular.otf,
  Scale=1.1
]

\setCJKmainfont{Noto Serif CJK SC}[
  BoldFont=Noto Serif CJK SC Bold,
  ItalicFont=FandolKai-Regular,
  Script=CJK Ideographic,Language=Chinese Simplified
]

% NOTE: FandolKai 没有 Script 选项

\setCJKsansfont{Noto Sans CJK SC DemiLight}[
  BoldFont=Noto Sans CJK SC Medium,
  Script=CJK Ideographic,Language=Chinese Simplified
]


% 暴力修正 FandolKai Fandol-FangSong 的基线
% via: https://github.com/clerkma/ptex-ng-dist/issues/5

\usepackage{etoolbox} % 提供 \appto

\makeatletter
\newcommand*\original@CJKsymbol{}
\newcommand*\original@CJKpunctsymbol{}
\let\original@CJKsymbol\CJKsymbol
\let\original@CJKpunctsymbol\CJKpunctsymbol
\newcommand*\raise@Fandol@CJK[1]{\raise0.08\ccwd\hbox{#1}}
\appto\itshape{%
  \let\CJKsymbol\raise@Fandol@CJK
  \let\CJKpunctsymbol\raise@Fandol@CJK
}
\appto\kaishu{%
  \let\CJKsymbol\raise@Fandol@CJK
  \let\CJKpunctsymbol\raise@Fandol@CJK
}
\appto\fangsong{%
  \let\CJKsymbol\raise@Fandol@CJK
  \let\CJKpunctsymbol\raise@Fandol@CJK
}
\appto\upshape{%
  \let\CJKsymbol\original@CJKsymbol
  \let\CJKpunctsymbol\original@CJKpunctsymbol
}
% 修正包裹在 Fandol 字体中的脚注 另见 latex.ltx 6403 行
\appto\reset@font{%
  \let\CJKsymbol\original@CJKsymbol
  \let\CJKpunctsymbol\original@CJKpunctsymbol
}
\makeatother

%----------------------------------------------------------------------------------------
% 标记的样式设定
%----------------------------------------------------------------------------------------

\newfontfamily\raintree{Raintree}[Scale=1.1] % 英文发布版带字体 注意版权问题

\newfontfamily\lmmonolightsmall{Latin Modern Mono Light 10 Regular}[Scale=1.1] % 拉大 1.1 倍适应中文正文
\newfontfamily\lmmonolightbig{Latin Modern Mono Light 10 Regular}[Scale=1.2] % 拉大 1.2 倍适应中文标题
\newfontfamily\lmmonocaps{Latin Modern Mono Caps 10 Regular}[Scale=1.1] % 拉大 1.2 倍适应中文

% mt ->  mechinetalk 需要区分中英文
% 中文:
\newcommand{\mtzh}{\lmmonolightsmall \fangsong} % \lmmonolight 1.2 倍放缩
\newcommand{\mtzhpr}[1]{{\mtzh #1}} % mtzhpr -> mtzh prefix
% 
% 英文:
\newcommand{\mten}{\raintree}
\newcommand{\mtenpr}[1]{{\mten #1}} % mtenpr -> mtzh prefix


\newcommand{\rt}{\sffamily} % rt -> radiotalk
\newcommand{\py}{\sffamily} % py -> 怕音

\newcommand{\rtpr}[1]{{\rt #1}} % rtpr -> radiotalk prefix
\newcommand{\pypr}[1]{{\py #1}} % pypr -> py prefix
\newcommand{\rcpr}[1]{\emph{#1}} % 回忆 mrpr -> recall prefix
\newcommand{\thpr}[1]{\emph{#1}} % 心理活动 thpr -> thought prefix
\newcommand{\stpr}[1]{{\itshape \CJKunderdot{#1}}} % 强调 st -> stress % 不使用粗体 使用加点 % FIXME: 处理不优雅需要改进
\newcommand{\rspr}[1]{\emph{#1}} % 录音声音 rspr -> record sound
\newcommand{\wrpr}[1]{{\itshape #1}} % 书写 wr -> write 

%----------------------------------------------------------------------------------------
%	环境标记
%----------------------------------------------------------------------------------------

% 题记样式
\newenvironment{intro}{\begin{center}\slshape \kaishu}
{\end{center}\medskip}

% \newenvironment{song}{\begin{flushright}\itshape \kaishu {\awesomefont  \quad}}{\end{flushright}}

% 唱歌样式
\newenvironment{song}{\begin{flushright}\itshape \kaishu}{\end{flushright}}

% 音乐样式
\newenvironment{music}{\begin{flushright}\itshape \kaishu {\awesomefont  }\hspace{\fill}}{\end{flushright}}

% \newenvironment{song}{\begin{flushright}\itshape \kaishu \marginpar[]{ \awesomefont }}{\end{flushright}}

% \newenvironment{song}{\marginpar[]{ \awesomefont }\begin{flushright}\itshape \kaishu}{\end{flushright}}


\newcommand\daytimeplace[4]{
  \begin{table}[H]
    \centering \fangsong \lmmonolightbig
    第 #1 天

    大约时间:#2

    地点:#3

    {\lmmonolightsmall {\lmmonocaps Loaction}: #4}
  \end{table}
}

\newcommand\unknowndaytimeplace{
  \begin{table}[H]
    \centering \fangsong \lmmonolightbig
    日期:N/A

    时间:N/A
    
    地点:N/A
  \end{table}
}

\newcommand\englishdaytimeplace[3]{
  \begin{table}[H]
    \centering \ttfamily
    \textsc{Day} #1

    \textsc{Time} approximately #2

    \textsc{Loaction}: #3
  \end{table}
}

\newcommand\englishunknowndaytimeplace{
  \begin{table}[H] % 钉死
    \centering
    \begin{tabular}{lr}
      \texttt{\textsc{Day}} & \texttt{N.A.} \\
      \texttt{\textsc{Time}} & \texttt{N.A.} \\
      \texttt{\textsc{Loaction}} & \texttt{N.A.} \\
    \end{tabular}    
  \end{table}
}

\font\eightssi=cmssqi8 % PostScript 字体

\newenvironment{motto}{
~\vfill
\begin{flushright}
  \eightssi \fangsong
}{\end{flushright}}

\newenvironment{note}{
\paragraph{尾注}}{\bigskip}

\newenvironment{engnote}{
\paragraph{Tailnote}}{\bigskip}

\newcommand\printstatus[7]{
\begin{flushleft}
\textbf{帕比的属性}
\end{flushleft}

\begin{itemize}
\item 力量:#1
\item 感知:#2
\item 耐力:#3
\item 魅力:#4
\item 智力:#5
\item 敏捷:#6
\item 幸运:#7
\end{itemize}
}

\newcommand\engprintstatus[7]{
\begin{flushleft}
\textbf{Puppy's S.P.E.C.I.A.L.}
\end{flushleft}

\begin{itemize}
\item{\makebox[2cm][l]{Strength:} #1}
\item{\makebox[2cm][l]{Perception:} #2}
\item{\makebox[2cm][l]{Endurance:} #3}
\item{\makebox[2cm][l]{Charisma:} #4}
\item{\makebox[2cm][l]{Intelligence:} #5}
\item{\makebox[2cm][l]{Agility:} #6}
\item{\makebox[2cm][l]{Luck:} #7}
\end{itemize}
}

%----------------------------------------------------------------------------------------
% 西文排版
%----------------------------------------------------------------------------------------

% \usepackage[english]{babel}
\usepackage{microtype} % 微排版

% 设置西文段落行距
% 参考:
% - [zhlineskip: 增加范例,单独设置西文段落的行距 · Issue #425 · CTeX-org/ctex-kit · GitHub](https://github.com/CTeX-org/ctex-kit/issues/425)
% - zhlineskip 宏包手册

\newcommand{\englinespread}{% 创建 \englinespread 标记
  \linespread{1.1} \selectfont % 西文整体行距调整
}

\usepackage[bodytextleadingratio=1.5]{zhlineskip} % 正文行高修正

\newenvironment{englishpar}  % 新建「英文段落」环境
{\addvspace\medskipamount}  % 段前间距,上文段落需结束
{\par\addvspace\medskipamount} % 段后间距

\newenvironment{englishlyric}{}{}  % 新建「英文歌词」环境

% zhlineskip 宏包命令
% \SetTextEnvironmentSinglespace{1} % 根据需要修改此数,默认值是 1,适合西文「单倍」行距

\RestoreTextEnvironmentLeading{englishpar}
\RestoreTextEnvironmentLeading{englishlyric}


%----------------------------------------------------------------------------------------
% 脚注样式
%----------------------------------------------------------------------------------------

\usepackage{xpatch} % 提供 \xpatchcmd

% 调整脚注的分割线 via:https://github.com/muzimuzhi/latex-examples/blob/master/footnote-chinese-style.tex https://zhuanlan.zhihu.com/p/74515148

\xpatchcmd\footnoterule
  {.4\columnwidth}
  {1in}
  {}{\fail}

% ----------


\newfontfamily\footnotefont{Noto Serif CJK SC}[Script=CJK Ideographic,Language=Chinese Simplified]
% \newCJKfontfamily[footnotecjk]\footnotecjkfont{Noto Serif CJK SC}[Script=CJK Ideographic,Language=Chinese Simplified]

\newcommand{\footnotespacefix}{\hspace{-0.5\ccwd}}

\usepackage{scrextend} % 提供 \deffootnote

% NOTE: 脚注大小 7.5bp 空两个字 2em
% NOTE: 直接使用中文字体

% 样式:[1]

% \renewcommand{\thefootnote}{{\footnotefont[\arabic{footnote}]}}
% \deffootnote[0em]{0em}{2\ccwd}{{\footnotesize\thefootnotemark\hspace{\ccwd}}}

% 样式:〔1〕(最好)

\renewcommand{\thefootnote}{{\footnotefont\makexeCJKinactive\hspace{-0.5\ccwd}〔\arabic{footnote}〕\hspace{-0.5\ccwd}\makexeCJKactive}}
\deffootnote[0em]{0em}{2\ccwd}{{\footnotesize\thefootnotemark\hspace{\ccwd}}}

% NOTE: 不用 em 因为英文字体放大了 1.1 倍

% 样式:[1]

% \renewcommand{\thefootnote}{{\footnotefont\makexeCJKinactive\hspace{-0.5\ccwd}[\arabic{footnote}]\hspace{-0.5\ccwd}\makexeCJKactive}}
% \deffootnote[0em]{0em}{2\ccwd}{{\footnotesize\thefootnotemark\hspace{\ccwd}}}

% FIXME: hyperref 给的方框过大

% 样式:带圈数字
% via: [带圈数字](https://stone-zeng.github.io/2019-02-09-circled-numbers)

% \usepackage{xunicode-addon}

% % XeLaTeX 下需要把全体带圈数字都设置成 Default 类
% \xeCJKDeclareCharClass{Default}{"24EA, "2460->"2473, "3251->"32BF}

% % 放置钩子,只让带圈字符才需更换字体

% \AtBeginUTFCommand[\textcircled]{\begingroup\footnotefont}
% \AtEndUTFCommand[\textcircled]{\endgroup}

% \renewcommand{\thefootnote}{\textcircled{\arabic{footnote}}}
% \deffootnote[0em]{0em}{2\ccwd}{\footnotesize\thefootnotemark\hspace{\ccwd}} % 需要留空


% NOTE: 以下也可行 via:https://github.com/muzimuzhi/latex-examples/blob/master/footnote-chinese-style.tex

% \makeatletter
% % add wrapper \textcircled,
% % adapted from latex.ltx, line 6380:
% \renewcommand\thefootnote{\textcircled{\@arabic\c@footnote}}

% % use separate footnote mark command
% \xpatchcmd\@makefntext
%   {\hb@xt@1.8em{\hss\@makefnmark}}
%   {\hb@xt@1.8em{\hss\@makefnmarkNormal}\space}
%   {}{\fail}

% % use non-suprescript style with lower baseline
% % adapted from latex.ltx, line 6383:
% % \def\@makefnmarkNormal{\lower .3ex \hbox{\normalfont\@thefnmark}} % NOTE: 不下调基线
% \def\@makefnmarkNormal{\hbox{\normalfont\@thefnmark}}

% \makeatother


%----------------------------------------------------------------------------------------
% 页眉样式
%----------------------------------------------------------------------------------------

\usepackage{fancyhdr}

\fancypagestyle{chinese}{%
\fancyhf{} % 清空

\fancyhead[LE,RO]{\thepage}

\fancyhead[LO]{\sffamily {\leftmark}} % 奇数页  \leftmark = 章节名称
\fancyhead[RE]{\sffamily {辐射小马国:粉色双眸}} % 偶数页
}

\fancypagestyle{english}{%
\fancyhf{} % 清空

\fancyhead[LE,RO]{\thepage}

\fancyhead[LO]{\sffamily {\leftmark}} % 奇数页  \leftmark = 章节名称
\fancyhead[RE]{\sffamily \uppercase{Fallout Equestria: Pink Eyes}} % 偶数页
}

%----------------------------------------------------------------------------------------
%	分割线样式
%----------------------------------------------------------------------------------------

% 第一种样式
% \newcommand{\horizonline}{
%     \begin{center}\rule{0.5\linewidth}{\linethickness}\end{center}
% }

% 第二种样式
% \newcommand{\horizonline}{
%     \begin{center}\includegraphics[width=0.5\linewidth]{image_line.png}\end{center}
% }

% 第三种样式 via: fancyhdr 宏包手册
\usepackage{fourier-orns}
\newcommand{\horizonline}{%
\begin{englishpar}
  \begin{center}
    \rule{0.2\linewidth}{\linethickness}
    \raisebox{-2.1pt}{\hspace{1em}\decofourleft\decotwo\decofourright\hspace{1em}}\,   % 然而并不对称 插入 \, 先修补一下
    \rule{0.2\linewidth}{\linethickness}
  \end{center}
\end{englishpar}
}

%----------------------------------------------------------------------------------------
%	数学相关
%----------------------------------------------------------------------------------------

\usepackage{amsmath,amssymb,amsthm}
% \usepackage{siunitx}

% NOTE: 使用统一风格的字体来表示

\newfontfamily\engfontapprox{FandolFang-Regular.otf}[Scale=1.1]

\newcommand{\engapprox}{{\engfontapprox ≈}}

%----------------------------------------------------------------------------------------
%	目录
%----------------------------------------------------------------------------------------

\setcounter{secnumdepth}{0} % 取消节编号

%----------------------------------------------------------------------------------------
%	LINKS & hyperref
%----------------------------------------------------------------------------------------

\usepackage{url}

% 强制 URL 换行 via: https://tex.stackexchange.com/questions/3033/forcing-linebreaks-in-url

\expandafter\def\expandafter\UrlBreaks\expandafter{\UrlBreaks%  save the current one
  \do\a\do\b\do\c\do\d\do\e\do\f\do\g\do\h\do\i\do\j%
  \do\k\do\l\do\m\do\n\do\o\do\p\do\q\do\r\do\s\do\t%
  \do\u\do\v\do\w\do\x\do\y\do\z\do\A\do\B\do\C\do\D%
  \do\E\do\F\do\G\do\H\do\I\do\J\do\K\do\L\do\M\do\N%
  \do\O\do\P\do\Q\do\R\do\S\do\T\do\U\do\V\do\W\do\X%
  \do\Y\do\Z}



\usepackage[xetex]{hyperref}

\hypersetup{breaklinks=true,
            pdfauthor={RandomQuantum},
            pdftitle={Fallout Equestria: Pink Eyes 辐射小马国:粉色双眸},
            pdfcreator={LaTeX with CTeX and hyperref},
            pdfkeywords={FoE}
            }

%----------------------------------------------------------------------------------------
% 目录问题解决
%----------------------------------------------------------------------------------------

% 需要放在 hyperref 后面
% 规定 chapter 的计数器的上级为 part,使得每个 part 后 chapter 清零
% 暴力使用 setcounter{chapter}{0} 会使得 hyperref 错误
\makeatletter
\@addtoreset{chapter}{part}
\makeatother
% via: [sectioning - How to reset chapter and section counter with \part - TeX - LaTeX Stack Exchange](https://tex.stackexchange.com/questions/54383/how-to-reset-chapter-and-section-counter-with-part)




